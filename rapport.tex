\documentclass[a4paper, oneside, 12pt]{article}

% Paquets pour le Français
\usepackage[utf8]{inputenc} % Gestion encodages
\usepackage[T1]{fontenc} % ???
\usepackage[francais]{babel} % Typographie française

\usepackage{graphicx}
\usepackage{here}

\usepackage[top=2.5cm, bottom=4cm, left=2.5cm, right=2.5cm, footskip=2cm]{geometry}

\newcommand\sectionSpeciale[1]{\addcontentsline{toc}{section}{#1}\section*{#1}}

\def\www{\emph{W3+}}
\def\siemens{\emph{Siemens}}

\author{ Lucien \sc{Guimier} (F5, \siemens) \\ Timothée \sc{Hamon} (F3, \www) }
\title{Dossier management}

\begin{document}

\maketitle
\thispagestyle{empty}

\newpage
\setcounter{page}{1}
\sectionSpeciale{Introduction}

\vfill

\tableofcontents

\vfill

\newpage
\section{Organisation des entreprises}

% TODO organigramme(s).
% TODO texte additionel…
% TODO Taille des équipes

\newpage
\section{Organisation des équipes}

\subsection{Organisation du travail}

\subsubsection{Organisation à court terme}

À \www, le travail est organisée selon la méthodologie \emph{Scrum} : les différents projets sont découpés en tâches courtes (une à huit heures) réparties sur une période de deux à quatre semaines. À la fin d’une telle période, le produit est en principe déployé chez le client. Chaque matin, l’équipe se réunit pour discuter de l’avancement des tâches et des éventuels problèmes rencontrés.

\begin{figure}[H]
	\centering
	\includegraphics[width=12cm]{img/scrum.png}
	\caption{Schéma de la méthodologie \emph{Scrum}}
\end{figure}

La résolution des tâches et l’association de celle-ci avec les modifications du code est suivies grâce à TFS (\emph{Team Foundation Server}).

\ 

La partie de l’équipe de \siemens\ où était intégrée Lucien {\sc Guimier} se réunit deux fois par semaine, le lundi et le mercredi, pour discuter de l’avancement des tâches. Le suivi de celles-ci est réalisé grâce à un tableau \emph{Kanban} (autre méthodologie associée à \emph{Agile}) : les tâches sont représentées par des cartons avec leur description, disposés dans des colonnes selon leur avancement (à faire, en cours, fini).

Un écran permet d’avoir un aperçu de l’avancement de la correction des problèmes et d’être informé des prochains jalons.

\ 

Les deux méthodologies sont assez similaires ; seule la fréquence de déploiement chez le client change : le logiciel est déployé chez \www\ au minimum une fois par mois, tandis que l’équipe de \siemens\ travaille sur la prochaine version majeure du logiciel développé, intégrant uniquement leurs modification à la version commune, la version majeure étant publiée moins d’une fois par an.

\newpage

\subsubsection{Organisation à long terme}

Tous les mois, l’équipe de \siemens\ (entière) organise un petit déjeuner de projet (« \textit{Projektfrühstück} »), durant lequel un ou deux membres de l’équipe présentent les nouveautés. Les présentations sont suivies en consommant un assortiment de jus, gâteaux, fromages, pains et charcuterie ; la consommation commence un peu avant, laissant le temps pour des discussions libres.

\subsubsection{Séminaires et formations}

Aucun séminaire ne semble organisé à \www, tandis qu’un moins un a eu lieu sur le complexe de \siemens\ pendant la période du stage.

\www\ fournit des formations à ses employés sur demande, à condition que celle-ci soit en concordance avec son domaine de travail et ses besoins. Elles sont en général réalisées par groupes.

% TODO
Nous n’avons pas d’informations sur les formations dispensées à \siemens.

\vfill

\subsection{Interactions entre collègues}

\subsubsection{Pauses pendant les horaires de travail}

Dans les deux entreprises, un espace est réservé pour les pauses café, chacun équipé d’une machine à café. Malgré les recommandations, l’équipe de \siemens\ dispose de plus d’une cafetière directement sur une table disposée entre les bureaux.

\ 

À \www, toute l’équipe est attendue à la pause du matin, vers 9$~$h —$~$avant la réunion journalière$~$— et une seconde pause est possible vers 14$~$h. Il arrive qu’un employé —$~$parfois plusieurs$~$— apportent des viennoiseries ou des gâteaux pour toute l’équipe en guise de petit déjeuner. Ces petits déjeuners ont en particulier lieu lors d’un anniversaire ou après un déploiement.

\ 

Du fait de la présence d’une cafetière au centre des postes de travail, l’équipe de \siemens\ n’a pas de telles pauses fixes, les cafés étant consommés directement pendant le travail. Les autres équipes cependant, semblent avoir des pauses communes à la salle café.

À plusieurs reprises, des goûters ou des petit déjeuners ont été organisés par certains employés à l’occasion d’événements particuliers comme un départ en voyage de noces ou des anniversaires.

\vfill

\newpage

\subsubsection{Pauses déjeuner}

\www\ ne disposant pas de cantine, les employés peuvent manger dans la salle de repos, équipée pour réchauffer des plats. Environ les deux tiers de ceux-ci l’utilisent quotidiennement. Environ une fois par mois, un repas collectif est organisé par les employés.

\ 

Le complexe de \siemens\ inclut deux cantines où les membres de l’équipe se rendent par groupes de quatre à huit personnes en général.

\ 

Dans les deux équipes, la pause dure environ une heure et les sujets de conversation concernent rarement le travail en cours.

\subsubsection{Rencontres hors travail}

Timothée {\sc Hamon} a participé à une pendaison de crémaillère chez un de ses collègues. À \siemens, conformément à l’habitude allemande, aucune rencontre au domicile d’un employé n’a eu lieu ; un repas a cependant été organisé à une brasserie proche et une randonnée avait lieu après le départ de Lucien {\sc Guimier}.

Sans y avoir participé, nous savons qu’un repas de Noël est organisé à \www\ et que d’autres équipes de \siemens\ organisent des balades en vélocipède.

\section{Gestion du stagiaire}

\newpage
\sectionSpeciale{Conclusion}

\end{document}
